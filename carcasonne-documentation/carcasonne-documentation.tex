\documentclass[11pt,a4paper]{article}
\usepackage[utf8]{inputenc}
\usepackage[german]{babel}
\usepackage[T1]{fontenc}
\usepackage{amsmath}
\usepackage{amsfonts}
\usepackage{amssymb}
\usepackage{graphicx}
\usepackage[margin=1.25cm]{geometry} % Puts the same margin on all borders of the document

% Packages

\usepackage{hyperref} % Generate hyperlinks to referenced items
\usepackage{adjustbox} % Used to change parameters in \includegraphics[scale=•]{•}
\usepackage{enumitem} % Provides several options for lists
\usepackage{verbatim} % Package to use \begin{comment}
\usepackage{pdfpages} % Used to import PDF pages
\usepackage{multirow} % Allows us to have a single cell in a table span multiple rows
\usepackage{makecell} % Allows us to format multiple lines in a single cell
\usepackage{listings} % Used to type code in \begin{lstlisting}
\usepackage{xcolor}  % Gives access to coloring text
\usepackage{longtable} % Allows us to create a table over multiple pages
\usepackage{float} % Improved placement of floating items



% Settings

\graphicspath{{./files/}} % Sets path for files to the files folder in the same directory

\hypersetup{
    colorlinks=false, %set true if you want colored links
    linktoc=all,     %set to all if you want both sections and subsections linked
    linkcolor=blue,  %choose some color if you want links to stand out
}


\begin{titlepage}
  \title{Dokumentation Projektarbeit Carcasonne} % document_name-type_of_document
  \author{Leopold Keller, Laurenz Kammeyer, Frederick Wichert, Jonas Milkovits}
  \date{}
\end{titlepage}

\lstset {
	literate={~} {$\sim$}{1},
    language=Java,
    basicstyle=\footnotesize,
    numbers=left,
    stepnumber=1,
    showstringspaces=false,
    tabsize=1,
    breaklines=true,
    breakatwhitespace=false,
    inputencoding=utf8,
    extendedchars=true   
}


\begin{document}

\pagenumbering{gobble}
\maketitle
\pagenumbering{roman} % i, ii, iii on beginning pages, that don't have content
\tableofcontents
\clearpage
\pagenumbering{arabic} % 1,2,3 on content pages


\section{Dokumentation - 6.1.4(b)}

\section{Dokumentation - 6.1.5}

\section{Dokumentation - 6.3.1}

Die graphische Oberfläche wurde um folgende Elemente erweitert:
\begin{itemize}
	\item  \textbf{Tabelle}
		\begin{itemize}
		\item \textbf{Ziel der Spaltengestaltung:}
			\begin{enumerate}
			\item Möglichst übersichtliche Gestaltung der Spalten
			\item kurze, aussagekräftige Namen
			\item Informationen so übersichtlich wie möglich $\rightarrow$ wenige Spalten
			\end{enumerate}
		\item Die Tabelle wurde mit folgendem Code erstellt:
		\begin{lstlisting}
			private JTable scoreTable;
			List<ScoreEntry> list = resources.getScoreEntries();
			Object[] column = { "Date", "Name", "Score" };
			Object[][] data = new Object[list.size()][3];
			scoreTable = new JTable(data, column);	
		\end{lstlisting} 
		
		\item Das zweidimensionale Array \texttt{data} wurde mit folgendem Code befüllt:
		\begin{lstlisting}
		for (int i = 0; i < list.size(); i++) {
			if (list.get(i) != null) {
				data[i][0] = list.get(i).getDate();
				data[i][1] = list.get(i).getName();
				data[i][2] = list.get(i).getScore();
			}
		}
		\end{lstlisting}
		
		\item \textbf{Problem:} Beim Standardmodell editierbare Zellen
			\begin{itemize}
				\item[$\Rightarrow$] \textbf{Lösung:} Eigenes Table-Modell
				\item[]
						Das Überschreiben der Methode \texttt{isCellEditable(int row, int column}) erlaubt es 
						uns das Editieren der Zellen zu verhindern. Das Überschreiben der zweiten Methode
						benötigen wir später für das Sortieren der Spalten.
			\end{itemize}
	
		\begin{lstlisting}
		DefaultTableModel tableModel = new DefaultTableModel(data, column) {

			@Override
			public boolean isCellEditable(int row, int column) {
				// all cells false
				return false;
			}

			public Class<?> getColumnClass(int columnIndex) {
				return getValueAt(0, columnIndex).getClass();
			}			

		};
		
		scoreTable.setModel(tableModel);
		\end{lstlisting}
		
		\item \textbf{Ergebnis:}		

		\item[]
		
		Wir haben eine Tabelle mit drei Spalten (Datum, Name, Punkte) erstellt.
		Das Datumsformat ist absichtlich kürzer gehalten, da unserer Meinung nach der Tag des Spiels ausreichend 		ist.
		\end{itemize}
		
	\pagebreak	
	
	\item \textbf{Scrollbar}
		\begin{itemize}
		\item Eine Scrollbar wird benötigt, falls zuviele Daten vorhanden sind
		\item Die Scrollbar wurde mit folgendem Code erstellt:
		\begin{lstlisting}
		private JScrollPane scrollPane;
		scrollPane = new JScrollPane(scoreTable);
		add(scrollPane);
		\end{lstlisting}
		\item Dies verknüpft die Tabelle mit der Scrollbar
		\end{itemize}
	
	\item \textbf{Sortierbarkeit der Spalten}
		\begin{itemize}
		\item[]
			 Die Sortierbarkeit der Spalten	wurde uns hier zwar nicht vorgegeben, war für uns aber ein wichtiges
			 Feature bei einer Highscore-Tabelle. Diese dient auch der einfacheren Verwendbarkeit für den
			 Endnutzer. Alle Spalten sind nach ihren jeweiligen Werten sortierbar.
		\item Dies wurde mit folgendem Code erreicht:
		\begin{lstlisting}
		TableRowSorter<TableModel> sorter = new TableRowSorter<TableModel>(scoreTable.getModel());
		scoreTable.setRowSorter(sorter);
		\end{lstlisting}
		\item[]
			Die Vorarbeit, die hierfür benötigt wurde, ist weiter oben in der Erstellung von 
			\texttt{DefaultTableModel} zu finden. Dies erlaubt es uns, auch das Datum sowie die Punktzahl zu
			sortieren. Ohne diese Vorarbeit könnte man das Ganze nur korrekt nach Namen sortieren, da der
			Autosorter den Wert des Strings vergleicht, was bei dem Datum und der Punktzahl zu falschen
			Sortierungen führt.
		
		
		\end{itemize}
	
	
\end{itemize}


\section{Dokumentation - 6.3.2}

\section{Dokumentation - 6.3.3}

\pagebreak

\section{Dokumentation - 6.3.4}
Der Computergegner besteht aus zwei Methoden:
\begin{itemize}

\item \textbf{\texttt{draw(GamePlay gp, Tile tile)}}
	\begin{itemize}
	
	\item[]
		Als ersten Schritt suchen wir uns alle möglichen Platziermöglichkeiten. \\
		Dies geschieht indem wir die Liste der platzierten Tiles des Boards durchlaufen und bei jedem Tile 
		alle Positionen (nördlich, südlich, westlich, östlich) überprüfen und diese, falls das zu 
		platzierende Tile anlegbar ist, abspeichern. \\
		Dieses Abspeichern geschieht in einer Liste einer selbst erstellten Klasse namens PointRotation, die 
		Koordinaten und die Rotation der Position abspeichern. Danach lassen wir uns eine zufällige Zahl 
		(zwischen 0 und der Länge Liste - 1) ausgeben und verwenden diese zufällige Position dann um an 
		dieser Stelle dann das zu platzierende Tile zu platzieren.
	
	\item Suche der möglichen Platzierungsmöglichkeiten:
	\begin{lstlisting}
	List<Tile> tiles = gc.getGameBoard().getTiles();
		List<PointRotation> possibleLocations = new ArrayList<PointRotation>();

		for (Tile t : tiles) {
			// check top
			for (int r = 0; r < 4; r++) {
				if (gc.getGameBoard().isTileAllowed(topTile, t.x, t.y - 1) && !	
				gc.getGameBoard().isTileAtPosition(t.x, t.y - 1)) {
					possibleLocations.add(new PointRotation(t.x, t.y - 1, topTile.getRotation()));
				}
				topTile.rotateRight();
			}

			// check right
			for (int r = 0; r < 4; r++) {
				if (gc.getGameBoard().isTileAllowed(topTile, t.x + 1, t.y) && !
				
				gc.getGameBoard().isTileAtPosition(t.x + 1, t.y)) {
					possibleLocations.add(new PointRotation(t.x + 1, t.y, topTile.getRotation()));
				}
				topTile.rotateRight();
			}

			// check left
			for (int r = 0; r < 4; r++) {
				if (gc.getGameBoard().isTileAllowed(topTile, t.x - 1, t.y) && !
				gc.getGameBoard().isTileAtPosition(t.x - 1, t.y)) {
					possibleLocations.add(new PointRotation(t.x - 1, t.y, topTile.getRotation()));
				}
				topTile.rotateRight();
			}

			// check bottom
			for (int r = 0; r < 4; r++) {
				if (gc.getGameBoard().isTileAllowed(topTile, t.x, t.y + 1) && !
				gc.getGameBoard().isTileAtPosition(t.x, t.y + 1)) {
					possibleLocations.add(new PointRotation(t.x, t.y + 1, topTile.getRotation()));
				}
				topTile.rotateRight();
			}
		}
	\end{lstlisting}
	
	\item Suche nach zufälliger Zahl:
	\begin{lstlisting}
		Random randomGen = new Random();
		int randomNum = randomGen.nextInt(possibleLocations.size());
		PointRotation randomLocation = possibleLocations.get(randomNum);
	\end{lstlisting}	
	
	\pagebreak	
	
	\item Rotation der zu platzierenden Tile und Platzieren dieser:
	\begin{lstlisting}
		// Rotate tile to be placed
		while (topTile.getRotation() != randomLocation.getRotation()) {
			topTile.rotateRight();
		}
		
		// Add rotated tile to GameBoard
		gc.getGameBoard().newTile(topTile, randomLocation.getX(), randomLocation.getY());
	\end{lstlisting}
	
	\end{itemize}
\item \textbf{\texttt{placeMeeple(GamePlay gp)}}
	\begin{itemize}
	
	\item[]
			Das Platzieren des Meeples läuft folgendermaßen ab. Wir speichern mithilfe der Methode 
			\texttt{getMeepleSpots()} alle möglichen MeepleSpots des zuletzt platzierten Tiles ab und nutzen 
			diese um uns eine ArrayList des Types Position zu erstellen. In dieser werden alle Positionen
			gespeichert, an denen ein Meeple platzierbar ist. Wir lassen uns dann wieder eine zufällige Zahl
			ausgeben (basierend auf der Länge der ArrayListe) und platzieren den Meeple an dieser Stelle,
			falls die Anzahl der Meeple größer als 0 ist. \\
			Falls es keinen verfügbaren MeepleSpot gibt, wird die Methode \texttt{nextRound()} aufgerufen,
			die die momentane Runde beendet.
			
	\item Überprüfung, ob MeepleSpot vorhanden ist:
	\begin{lstlisting}
		if (meepleSpots == null) {
			gp.nextRound();
			return;
		}
	\end{lstlisting}
	
	\item Abspeichern aller möglichen MeepleSpots:
	\begin{lstlisting}
		// boolean array of length 9, true where meeple can be placed on current tile
		boolean[] meepleSpots = gc.getGameBoard().getMeepleSpots();
		
		// array with all positions
		Position[] positions = Position.getAllPosition();
		
		// Puts all positions with a valid meeple spot into an ArrayList
		ArrayList<Position> possibleMeepleSpots = new ArrayList<Position>();
		
		for (int i = 0; i < positions.length; i++) {
			if (meepleSpots[i] == true) {
				possibleMeepleSpots.add(positions[i]);
			}
		}
	\end{lstlisting}
	
	\item Zufälliger MeepleSpot und Platzieren des Meeples:
	\begin{lstlisting}
		// Getting random meeple spot
		Random randomGen = new Random();
		int randomNum = randomGen.nextInt(possibleMeepleSpots.size());
		
		Position randomMeepleSpot = possibleMeepleSpots.get(randomNum);
		
		if(meeples > 0) {
			gp.placeMeeple(randomMeepleSpot);
		}
		else {
			gp.nextRound();
		}
	\end{lstlisting}
	
	\end{itemize}

\end{itemize}



\end{document}
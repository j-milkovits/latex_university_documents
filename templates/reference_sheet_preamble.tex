\documentclass[11pt,a4paper]{article}
\usepackage[utf8]{inputenc}
\usepackage[german]{babel}
\usepackage[T1]{fontenc}
\usepackage{amsmath}
\usepackage{amsfonts}
\usepackage{amssymb}
\usepackage{graphicx}
\usepackage[margin=1.25cm]{geometry} % Puts the same margin on all borders of the document

% Packages

\usepackage{hyperref} % Generate hyperlinks to referenced items
\usepackage{adjustbox} % Used to change parameters in \includegraphics[scale=•]{•}
\usepackage{enumitem} % Provides several options for lists
\usepackage{verbatim} % Package to use \begin{comment}
\usepackage{pdfpages} % Used to import PDF pages
\usepackage{multirow} % Allows us to have a single cell in a table span multiple rows
\usepackage{makecell} % Allows us to format multiple lines in a single cell
\usepackage{xcolor}  % Gives access to coloring text
\usepackage{longtable} % Allows us to create a table over multiple pages
\usepackage{float} % Improved placement of floating items
\usepackage{pdfpages} % Used to import pdf pages
\usepackage{booktabs} % Used for horizontal lines instead of \hline
\usepackage{siunitx} % Used to type the %-sign in text without it fucking my bracket color \SI{70}{\percent}
\usepackage{tabularx} % Tables with better adjustable widths
%\usepackage[none]{hyphenat} % Used to globally disable hyphenation
\usepackage[outputdir=build]{minted} % Used to syntax highlight code
\usepackage{tcolorbox} % Useful for making boxes, compatible with mined
% Settings

\graphicspath{{./files/}} % Sets path for files to the files folder in the same directory

\hypersetup{
    colorlinks=false, %set true if you want colored links
    linktoc=all,     %set to all if you want both sections and subsections linked
    linkcolor=blue,  %choose some color if you want links to stand out
}

% Some settings for itemize
\newcommand\sbullet[1][.5]{\mathbin{\vcenter{\hbox{\scalebox{#1}{$\bullet$}}}}} % creates a scalable bullet

\renewcommand\labelitemi{$\bullet$}             % Used to standardize symbols for itemize
\renewcommand\labelitemii{$\sbullet[.80]$}
\renewcommand\labelitemiii{$\sbullet[.65]$}
\renewcommand\labelitemiv{$\sbullet$}


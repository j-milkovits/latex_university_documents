\documentclass[11pt,a4paper]{article}
\usepackage[utf8]{inputenc}
\usepackage[german]{babel}
\usepackage[T1]{fontenc}
\usepackage{amsmath}
\usepackage{amsfonts}
\usepackage{amssymb}
\usepackage{graphicx}
\usepackage[margin=1.25cm]{geometry} % Puts the same margin on all borders of the document

% Packages

\usepackage{hyperref} % Generate hyperlinks to referenced items
\usepackage{adjustbox} % Used to change parameters in \includegraphics[scale=•]{•}
\usepackage{enumitem} % Provides several options for lists
\usepackage{verbatim} % Package to use \begin{comment}
\usepackage{pdfpages} % Used to import PDF pages
\usepackage{multirow} % Allows us to have a single cell in a table span multiple rows
\usepackage{makecell} % Allows us to format multiple lines in a single cell
\usepackage{listings} % Used to type code in \begin{lstlisting}
\usepackage{xcolor}  % Gives access to coloring text
\usepackage{longtable} % Allows us to create a table over multiple pages
\usepackage{float} % Improved placement of floating items



% Settings

\graphicspath{{./files/}} % Sets path for files to the files folder in the same directory

\hypersetup{
    colorlinks=false, %set true if you want colored links
    linktoc=all,     %set to all if you want both sections and subsections linked
    linkcolor=blue,  %choose some color if you want links to stand out
}

\begin{titlepage}
  \title{Titel hier einfügen} % document_name-type_of_document
  \author{Jonas Milkovits}
  \date{Last Edited: \today}
\end{titlepage}

\begin{document}

\pagenumbering{gobble}
\maketitle
\pagenumbering{roman} % i, ii, iii on beginning pages, that don't have content
\tableofcontents
\clearpage
\pagenumbering{arabic} % 1,2,3 on content pages


\section{Einleitung}
\begin{itemize}
  \item \textbf{Problem} im Sinne der Informatik
    \begin{itemize}
      \item Enthält eine Beschreibung der Eingabe 
      \item Enthält eine Beschreibung der Ausgabe
      \item Gibt \textbf{keinen} Übergang von Eingabe und Ausgabe an
      \item z.B.: Finde den kürzesten Weg zwischen zwei Orten
    \end{itemize}
  \item Probleminstanzen
    \begin{itemize}
      \item \textbf{Probleminstanz} ist eine konkrete Eingabenbelegung, für die entsprechende Ausgabe gewünscht ist
      \item z.B.: Was ist der kürzeste Weg vom Audimax in die Mensa?
    \end{itemize}
  \item Begriff des Algorithmus
    \begin{itemize}
      \item Endliche Folge von Rechenschritten, der eine \textbf{Ausgabe} in eine \textbf{Eingabe} verwandelt 
    \end{itemize}
  \item Anforderungen an Algorithmen
    \begin{itemize}
      \item \textbf{Spezifizierung} der Eingabe und Ausgabe 
        \begin{itemize}
          \item Anzahl und Typen aller Elemente ist definiert

        \end{itemize}
    \end{itemize}
  

\end{itemize}





\end{document}
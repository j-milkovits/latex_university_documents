\documentclass[11pt,a4paper]{article}
\usepackage[utf8]{inputenc}
\usepackage[german]{babel}
\usepackage[T1]{fontenc}
\usepackage{amsmath}
\usepackage{amsfonts}
\usepackage{amssymb}
\usepackage{graphicx}
\usepackage[margin=1.25cm]{geometry} % Puts the same margin on all borders of the document

% Packages

\usepackage{hyperref} % Generate hyperlinks to referenced items
\usepackage{adjustbox} % Used to change parameters in \includegraphics[scale=•]{•}
\usepackage{enumitem} % Provides several options for lists
\usepackage{verbatim} % Package to use \begin{comment}
\usepackage{pdfpages} % Used to import PDF pages
\usepackage{multirow} % Allows us to have a single cell in a table span multiple rows
\usepackage{makecell} % Allows us to format multiple lines in a single cell
\usepackage{listings} % Used to type code in \begin{lstlisting}
\usepackage{xcolor}  % Gives access to coloring text
\usepackage{longtable} % Allows us to create a table over multiple pages
\usepackage{float} % Improved placement of floating items



% Settings

\graphicspath{{./files/}} % Sets path for files to the files folder in the same directory

\hypersetup{
    colorlinks=false, %set true if you want colored links
    linktoc=all,     %set to all if you want both sections and subsections linked
    linkcolor=blue,  %choose some color if you want links to stand out
}

\begin{titlepage}
  \title{Algorithmen und Datenstrukturen} % document_name-type_of_document
  \author{Jonas Milkovits}
  \date{Last Edited: \today}
\end{titlepage}

\begin{document}

\pagenumbering{gobble}
\maketitle
\pagenumbering{roman} % i, ii, iii on beginning pages, that don't have content
\tableofcontents
\clearpage
\pagenumbering{arabic} % 1,2,3 on content pages


\section{Einleitung}

\paragraph{Problem im Sinne der Informatik}
	\begin{itemize}
      	\item Enthält eine Beschreibung der Eingabe 
      	\item Enthält eine Beschreibung der Ausgabe
      	\item Gibt \textbf{keinen} Übergang von Eingabe und Ausgabe an
      	\item z.B.: Finde den kürzesten Weg zwischen zwei Orten
	\end{itemize}
	
\paragraph{Probleminstanzen}
    \begin{itemize}
      	\item Probleminstanz ist eine konkrete Eingabenbelegung, für die entsprechende Ausgabe gewünscht ist
      	\item z.B.: Was ist der kürzeste Weg vom Audimax in die Mensa?
	\end{itemize}
	
\paragraph{Begriff des Algorithmus}
    \begin{itemize}
    	\item Endliche Folge von Rechenschritten, der eine Ausgabe in eine Eingabe verwandelt 
	\end{itemize}
	
\paragraph{Anforderungen an Algorithmen}
    \begin{itemize}
      	\item Spezifizierung der Eingabe und Ausgabe 
        	\begin{itemize}
          		\item Anzahl und Typen aller Elemente ist definiert
        	\end{itemize}
		\item Eindeutigkeit
        	\begin{itemize}
          		\item Jeder Einzelschritt ist klar definiert und ausführbar 
          		\item Die Reihenfolge der Einzelschritte ist festgelegt
        	\end{itemize}
      	\item Eindlichkeit
        	\begin{itemize}
          		\item Notation hat eine endliche Länge
        	\end{itemize}
	\end{itemize}

\paragraph{Eigenschaften von Algorithmen}
	\begin{itemize}
		\item Determiniertheit
			\begin{itemize}
				\item Für gleiche Eingabe stets die gleiche Ausgabe (andere mögliche Zwischenzustände)
			\end{itemize}
		\item Determinismus
			\begin{itemize}
				\item Für gleiche Eingabe stets identische Ausführung und Ausgabe
			\end{itemize}
		\item Terminierung
			\begin{itemize}
				\item Algorithmus läuft für jede Eingabe nur endlich lange
			\end{itemize}
		\item Korrektheit
			\begin{itemize}
				\item Algorithmus berechnet stets die spezifizierte Ausgabe (falls dieser terminiert)
			\end{itemize}
		\item Effizienz
			\begin{itemize}
				\item Sparsamkeit im Ressourcenverbrauch (Zeit, Speicher, Energie,...)
			\end{itemize}
	\end{itemize}

\paragraph{Begriff der Datenstrukturen}
	\begin{itemize}
		\item Methode, um Datenabzuspeichern und zu organisieren
		\item Erleichtert Zugriff auf Daten und Modifikation der Daten
		\item Beinhalten Strukturbestandteile und Nutzerdaten (Payload)
		\item Sequenzen: Arrays, Listen,...
		\item Topologische Strukturen: Bäume, Graphen,... 
    \end{itemize}
    
\section{Sortieren}

\subsection{Allgemeine Informationen}

\paragraph{Das Sortierproblem} 
    \begin{itemize}
        \item Ausgangspunkt: Folge von Datensätzen $D_1,D_2,...,D_n$
        \item Zu sortierende Elemente heißen auch Schlüssel(werte)
        \item Ziel: Datensätze so anzuordnen, dass die Schlüsselwerte sukzessive ansteigen/absteigen
        \item Bedingung: Schlüsselwerte müssen vergleichbar sein
        \item Durchführung:
            \begin{itemize}
                \item Eingabe: Sequenz von Schlüsselwerten $<a_1,a_2,...,a_n>$
                \item Engabe ist eine \textbf{Instanz} des Sortierproblems
                \item Ausgabe: Permutation $<a_1', a_2', ..., a_n'>$ derselben Folge mit Eigenschaft $a_1' \leq ... \leq a_n'$
            \end{itemize}
        \item Algorithmus \textbf{korrekt}, wenn dieser das Problem für alle Instanzen löst
    \end{itemize}

\paragraph{Exkurs: Totale Ordnung}
    \begin{itemize}
        \item Sei $M$ eine nicht leere Menge und $\leq \subseteq MxM$ eine binäre Relation auf $M$
        \item Das Paar $(M, \leq)$ heißt genau dann totale Relation auf der Menge $M$, wenn Folgendes erfüllt ist:
            \begin{itemize}
                \item Reflexivität: $\forall x \in M: x \leq x$
                \item Transitivität: $\forall x,y,z \in M: x \leq y \land y \leq z \Rightarrow x \leq z$
                \item Antisymmetrie: $\forall x,y \in M: x \leq y \land y \leq x \Rightarrow x = y$
                \item Totalität: $\forall x,y \in M: x \leq y \lor y \leq x$
            \end{itemize}
        \item z.B.: $\leq$ Ordnung auf natürlichen Zahlen bildet eine totale Ordnung ($1 \leq 2 \leq 3 ...$)
        \item z.B.: Lexikographische Ordnung $\leq_{lex}$ ist eine totale Ordnung ($A \leq B \leq C ...$)
    \end{itemize}

\paragraph{Vergleichskriterien von Sortieralgorithmen}
    \begin{itemize}
        \item Berechnungsaufwand $O(n)$
        \item Effizient: Best Case vs Average Case vs Worst Case
        \item Speicherbedarf:
            \begin{itemize}
                \item in-place (in situ): Zusätzlicher Speicher von der Eingabegröße unabhängig
                \item out-of-place: Speichermehrbedarf von Eingabegröße abhängig
            \end{itemize}
        \item Stabilität: Stabile Verfahren verändern die Reihenfolge von äquivalenten Elementen nicht
        \item Anwendung als Auswahlfaktor: 
            \begin{itemize}
                \item Hauptoperationen beim Sortieren: Vergleiche und Vertausche
                \item Diese Operationen können sehr teuer oder sehr günstig sein, je nach Aufwand
                \item Anpassung des Verfahrens abhängig von dem Aufwand dieser Operationen
            \end{itemize}    
    \end{itemize}

\subsection{Insertion Sort}

\paragraph{Idee}

    \begin{itemize}
        \item Halte die linke Teilfolge sortiert
        \item Füge nächsten Schlüsselwert hinzu, indem es an die korrekte Position eingefügt wird
        \item Wiederhole den Vorgang bis Teilfolge aus der gesamten Liste besteht
    \end{itemize}

\paragraph{Code}
    \begin{minted}[autogobble,escapeinside=||]{c}  
    1 FOR j = 1 TO A.length - 1
    2   key = A[j]
    3   // Füge A[j] in die sortierte Sequenz A[0...j-1] ein
    4   i = j - 1
    5   WHILE i |$\geq$| 0 and A[i] > key
    6       A[i + 1] = A[i]
    7       i = i - 1
    8   A[i + 1] = key
    \end{minted}



\section{Pseudocode in der Vorlesung AuD}

\paragraph{Datentypen}
	\begin{itemize}
		\item String
			\begin{itemize}
				\item Aufbau: \mint{c}|"Die Summe ist"|
				\item Konkatenation: \mint{c}|"Die Summe ist" summe|  
            \end{itemize}
        \item Array
            \begin{itemize}
                \item \texttt{A}: Bezeichung eines Arrays \texttt{A}
                \item \texttt{A[i]} Zugriff auf \texttt{(i+1)}-tes Element des Arrays
            \end{itemize}
	\end{itemize}


\paragraph{Methoden}
     \begin{itemize}
	  	\item Rückgabe:
	  		\mint{c}|return summe| 
    \end{itemize}

\paragraph{Schleifen}
  	\begin{itemize}
		\item While-Schleife 
        \item[]
            \begin{minted}[autogobble]{c}  
                WHILE summe <= n
                    summe = summe + 1
                ENDWHILE
            \end{minted}
      \end{itemize}
      
\paragraph{Variablen}
	\begin{itemize}
	 	\item Initialisierung
        \item[] \mint{c}|summe := 0| 
            
	
	\end{itemize}

  






\end{document}